% Preamble für Uni-Mitschriften
% Stark inspiriert von https://github.com/gillescastel/university-setup/blob/master/preamble.tex
%  ┌────────┐
%  │ Basics │
%  └────────┘
\author{Maximilian Moeller} 
% Umlaute
\usepackage[utf8]{inputenc}
% westeuropäische Codierung (verbessert Trennung und copy/paste)
\usepackage[T1]{fontenc}
% spezielle Sonderzeichen
\usepackage{textcomp}
% deutsches Sprachpaket
\usepackage[ngerman, english]{babel}
\usepackage{url}
% Einbindung von Bildern
\usepackage{graphicx}
% Bilder auf der Titelseite
\usepackage{titlepic}
% verbesserte floating-Objekte (z.B. figures/tables)
\usepackage{float}
% verbesserter Schriftsatz (nicht sichtbare Abweichungen)
\usepackage[]{microtype}
% verbesserte Tabellen
\usepackage{booktabs}
% Kontrolle über enumerate, itemize und description
\usepackage{enumitem}
% PDF-Version 1.7
\pdfminorversion=7
% Platz zwischen Paragraphen
\usepackage{parskip}
% Keine Zeilennummern auf leeren Seiten
\usepackage{emptypage}
\usepackage{subcaption}
% Text in mehreren Spalten
\usepackage{multicol}
% Mehr Farben
\usepackage{xcolor}
% Lorem ipsum mit \blindtext
\usepackage[]{blindtext}
% Ersetzt etwas durch nichts (debugging)
\newcommand\hide[1]{}

% ┌────────────┐
% │ Mathematik │
% └────────────┘
% mathematische Pakete
\usepackage{amsmath, amsfonts, mathtools, amsthm, amssymb, mathrsfs}
% Terme streichen (optional mit Wert, z.B. Wert -> 0)
\usepackage{cancel}
% Bold math
\usepackage{bm}
% Shortcuts für Zahlenbereiche
\renewcommand\O{\ensuremath{\emptyset}}
\newcommand\N{\ensuremath{\mathbb{N}}}
\newcommand\Z{\ensuremath{\mathbb{Z}}}
\newcommand\Q{\ensuremath{\mathbb{Q}}}
\newcommand\R{\ensuremath{\mathbb{R}}}
\newcommand\C{\ensuremath{\mathbb{C}}}
% Funktionen und variablen aus mehreren Buchstaben
\newcommand{\var}[1]{\mathit{#1}}
\newcommand{\func}[1]{\operatorname{#1}}

% Mehr Sonderzeichen (z.B. Blitz oder eckige Doppelklammern)
\usepackage{stmaryrd}
% Widerspruch als Blitz
\newcommand\contra{\scalebox{1.5}{$\lightning$}}
\newcommand\defeq{\stackrel{\text{def}}{=}}

% SI-Einheiten
\usepackage{siunitx}  
\sisetup{locale = DE}

% ┌──────────────┐
% │ Environments │
% └──────────────┘
% Kästen um Definitionen etc.
\usepackage{mdframed}
% Layout der Kästen
\mdfsetup{skipabove=1em,skipbelow=0em}
\theoremstyle{definition}
\newmdtheoremenv[nobreak=true]{definition}{Definition}[section]
\newmdtheoremenv[nobreak=true]{theorem}[definition]{Theorem}
\newmdtheoremenv[nobreak=true]{prop}[definition]{Proposition}
\newmdtheoremenv[nobreak=true]{lemma}[definition]{Lemma}
\newmdtheoremenv[nobreak=true]{corollary}[definition]{Corollary}
\newmdtheoremenv[nobreak=true]{axiom}{Axiom}
\newtheorem*{example}{Example}
\surroundwithmdframed[topline=false, bottomline=false, rightline=false, linewidth=4pt, linecolor=gray, backgroundcolor=lightgray]{example}
\newtheorem*{notation}{Notation}
\newtheorem*{note}{Note}
\newtheorem*{problem}{Problem}

% deutsche Titel
\newmdenv[topline=false, bottomline=false, rightline=false, linewidth=4pt, linecolor=gray, backgroundcolor=lightgray, frametitle={Beispiel:}]{beispiel}
\newmdenv[topline=false, bottomline=false, rightline=false, linewidth=2pt, linecolor=gray, frametitle={Merke:}]{merke}

% \lecture starts a new lecture
% Usage:
% \lecture{1}{Freitag, 16.04.}{Einleitung}
% xifthen to adjust @lecture based on content of #3
\usepackage{xifthen}
% marginnote to make notes on the right side of the document
\usepackage[]{marginnote}
\makeatother
\def\@lecture{}%
\newcounter{lecture}
\newcommand{\lecture}[3]{
	\stepcounter{lecture}

    \ifthenelse{\isempty{#3}}{%
        \def\@lecture{VL #1}%
		\marginnote{{VL #1,\\ #2}}
    }{%
		\def\@lecture{VL #1: #3}%
		\marginnote{{VL #1, #3, #2}}
    }%
}

% Kopf- und Fußzeilen
\usepackage{fancyhdr}
\pagestyle{fancyplain}
\fancyhf{}
\renewcommand{\chaptermark}[1]{\markboth{#1}{}}
\fancyhead[R]{\@lecture}
\fancyhead[L]{\chaptername\ \thechapter : \leftmark}
\fancyfoot[C]{\thepage}
\renewcommand{\headrulewidth}{0.4pt}
\renewcommand{\footrulewidth}{0.4pt}
% nicht auf Titelseiten von Kapiteln
\fancypagestyle{plain}{
  \fancyhf{}
  \fancyfoot[C]{\thepage}
  \renewcommand{\headrulewidth}{0pt}
  \renewcommand{\footrulewidth}{0.4pt}
}
\makeatletter

% Fix some spacing
% http://tex.stackexchange.com/questions/22119/how-can-i-change-the-spacing-before-theorems-with-amsthm
\def\thm@space@setup{%
  \thm@preskip=\parskip \thm@postskip=0pt
}
\makeatother

% ┌──────┐
% │ tikz │
% └──────┘
\usepackage{tikz}
\usetikzlibrary{arrows, automata}
\tikzset{
	->, % directed edges
	>=stealth, % bold arrow heads
	node distance = 2cm,
	default/.style={circle, thin, draw=black, fill=gray!25}
}
